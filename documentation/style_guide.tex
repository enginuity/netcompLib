\documentclass[11pt]{article}
\usepackage{graphicx}
\usepackage[margin=1in]{geometry}


\newcommand{\codefilestatus}[3]{\begin{itemize}
\item Last Updated: #1
\item Documented? #2
\item Standardized Parameterization? #3
\end{itemize}}

\begin{document}


\title{Documentation for netcomplib}

\maketitle
\tableofcontents
\pagebreak


\section{Standard Function Parameter Names}
Tables for the appropriate (consistent) parameter names

\subsection{General Parameters}
\begin{tabular}{|l|r|}
\hline
{\bf Object Type} & {\bf Parameter Names} \\


\hline
Num. Nodes in network & Nnodes \\

\hline
Num. Simulations & Nsim \\

\hline
Num. Network observation (pairs) & Nobs \\

\hline 
General Logging File	 & logfile \\

\hline
\end{tabular}



\subsection{Adjacency \& Probability Matrices}
\begin{tabular}{|l|r|}
\hline
{\bf Object Type} & {\bf Parameter Names} \\


\hline
Observed adjacency matrix (matrix) & obsadjm; obsadjm[k]; adjm[k];\\

\hline 
Observed adjacency matrices (array) & obsadjs; adja[k]; \\

\hline
Edge Probability Matrix (matrix) & epmat \\

\hline
\end{tabular}

\subsection{General Models}
\begin{tabular}{|l|r|}
\hline
{\bf Object Type} & {\bf Parameter Names} \\


\hline
HRG Model & tree; tree[k]; treel \\
\hline
Baseline (generating) model & btree; btree[k]; btreel \\
\hline
Fixed (estimating) model & ftree; ftree[k]; ftreel \\

\hline
\end{tabular}


\section{Object Types}
The following will be a description of different types of objects needed for different model classes and such
\subsection{Hierarchical Random Graphs (HRG)}
HRG models need to have the tree information stored: This is stored as a list with the following fields: 
\begin{itemize}
\item \$prob = Node probabilities
\item \$children = Children vector
\item \$parents = Parent vector 
\item \$nodes = Number of nodes in the network model
\end{itemize}

\subsection{General Network Model}
\subsubsection{Old Format}
There is an old model format that is intended to store lists of models. The format is a list with: 
\begin{itemize}


\item \$mode = "tree", "block", or "latent"
\item \$m1, \$m2 = individual model lists that one can generate observations from. 
\end{itemize}

\subsubsection{Generating Models}
A list of generating models should be stored in a variable like `gen\_models', and looks like: 
\begin{itemize}
\item \$mode = Type of generating model: tree, block, latent
\item \$m1, \$m2 = Models to run \_network functions on. 
\item \$is\_null = null hypothesis true? (m1 = m2)
\item \$Nnodes = number of nodes in network 
\end{itemize}

\subsubsection{Fitting Models}
A list of such fitting models: This format leads to faster processing. Each list looks like: stored in variables like `fit\_models'
\begin{itemize}


\item \$mode = tree, block, random
\item \$model\_list = list of models
\item \$Nnodes  = number of nodes in network
\item \$Nmodels = number of models in model\_list
\end{itemize}

\subsection{Parameter Lists}
\subsubsection{Parameter List for Simulations}
\begin{itemize}
\item thres\_ignore = Edge groups with fewer than this many cells will be ignored
\item cc\_adj = Number of 'sd's away to be conservative with, regarding correlation estimate. 
\item alphas = Sizes of test to generate cutoffs for
\item n\_models = Number of fixed edge partitions to use

\end{itemize}


\section{Model Generation and Network Simulation}

\subsection{model\_generation\_cases.R}
Contains code to sample tree, block, latent models, and to sample networks from these models. 

\section{Hypothesis Test and Simulations}
\subsection{compute\_lrt\_subfxs.R}
This contains the newer code for fast (in R) simulations. 

\section{Unsorted Function Files}

\subsection{allcode-general.R -- TO CLEAR UP}
Collection of main tree-processing code. This should be split up into sections...

\subsection{bootstrap\_networkcomp.R -- TO CLEAR UP}
Stores the bootstrap network comparison code. There are several versions, and they need to be combined into one... (TODO)

\subsection{compute\_edgep\_dist.R}
Functions to compute various distance measures

\subsection{compute\_lrt.R}
Functions to compute the likelihood ratio test



\subsection{distance\_distrib\_samp.R -- TO CLEAR}
Collection of more recent functions. Need to be organized. 


\subsection{model\_generation.R}
Functions to generate general models

\subsection{model\_generation\_hrg.R}
Functions to generate HRG models \\
This is to be obseleted

\subsection{model\_generation\_latent.R}
Functions to generate Latent space models \\
This is to be obseleted
\subsection{model\_generation\_sbm.R}
Functions to generate SBMs \\
This is to be obseleted

\subsection{model\_mcmc\_hrg.R}
MCMC code in R for the HRG

\subsection{multiple\_tree\_functions.R -- TO CLEAR}
Collection of functions to work with ensembles of data. 

\subsection{multiple\_testing\_fx.R}
Functions to perform / simulate multiple testing

\subsection{new\_sim\_fxs.R}
New work on simulation functions (this file is temporary)

\subsection{sim\_bootstrap\_dist.R}
Simulation functions for distances using bootstrap method

\subsection{sim\_lrt.R}
Simulation functions for the LRT method

\subsection{vis\_heatmap.R}
Code for visualization for a heatmap of probabilities

\pagebreak
\section{New Classes}
\subsection{NetworkModel}
\subsubsection{Hierarchy}
\begin{itemize}
\item NetworkModel (Abbreviated NetM)\\
FIELD -- Nnodes : numeric
\begin{itemize}
\item NetworkModelSBM\\
FIELD -- assign : numeric \\
FIELD -- probmat : matrix\\
\item NetworkModelHRG\\
FIELD -- parents : numeric\\
FIELD -- children : list\\
FIELD -- prob : numeric\\
\item NetworkModelLSM\\
FIELD -- locs : matrix\\
FIELD -- alpha : numeric\\
\item NetworkModelRND\\
FIELD -- counts : numeric\\
FIELD -- prob : numeric\\
FIELD -- ids : list\\
\item NetworkModelPair\\
FIELD -- m1 : NetworkModel\\
FIELD -- m2 : NetworkModel\\
FIELD -- is\_null : logical\\
\end{itemize}
\end{itemize}

\subsubsection{Functions}
\begin{itemize} 
\item getNnodes(NetM)\\
This returns the number of nodes in the network
\item getNetType(NetM)\\
This returns the type of network model
\item getEdgeProbMat(NetM, mode))\\
This computes the edge probability matrix\\
Mode = `prob' or `group', will return either a probability matrix or a matrix with ids for group assignment. 
\item sampleNetwork(NetM, Nobs, Nsim)\\
This samples networks from the model
\item extractStruct(NetM)
This converts the network model into a structure object (basically, throws out probability information)
\item computeLik(NetM, adja, loglik = TRUE)\\
This computes the likelihood of the adjacency matrix given the network model. loglik specifies whether we want the log-likelihood or the actual likelihood
\item computeKLdist(x, y = NULL) \\
Not implemented yet -- Computes the KL distance between two network models (x,y), or if x is a pair of network models. 
\end{itemize}

\subsection{NetworkStruct}
\subsubsection{Hierarchy}
\begin{itemize}

\item NetworkStruct (Abbreviated NetS)\\
FIELD -- Nnodes : numeric
\begin{itemize}
\item NetworkStructSBM\\
FIELD -- groups : numeric\\
FIELD -- counts : numeric\\
FIELD -- expand : list\\
FIELD -- correct : numeric\\
\item NetworkStructHRG\\
FIELD -- tree\_list : list\\
FIELD -- expand : list\\
FIELD -- counts : numeric\\
\item NetworkStructRND\\
FIELD -- counts : numeric\\
FIELD -- ids : list\\
\item NetworkStructList\\
FIELD -- models : list\\
\end{itemize}
\end{itemize}


\subsubsection{Functions}
\begin{itemize}
\item getNnodes(Net)
\item getNetType(Net)
\item fitModel(NetS, adja)\\
Not implemented -- This fits the model to the adjacency matrix/array, and returns the best fit (MLE) network model. 
\item computePval(NetS, adja1, adja2, Nobs, pl)\\
Returns a matrix of p-values (based on the parameter list) of fitting the model on the given NetS (or, list of NetS's, in which case a list of matrices is returned)
\item computePval\_fast(...)\\
Not implemented -- perhaps write a faster version of this (using global variables, and thus only callable in a very specific manner), as the current code might be slow (due to R's pass by copy as opposed to reference) -- this should end up being written in C++... 
\end{itemize}
\end{document}

