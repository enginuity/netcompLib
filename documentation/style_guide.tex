\documentclass[11pt]{article}
\usepackage{graphicx}
\usepackage[margin=1in]{geometry}
\usepackage{todonotes}

\newcommand{\codefilestatus}[3]{\begin{itemize}
\item Last Updated: #1
\item Documented? #2
\item Standardized Parameterization? #3
\end{itemize}}

\begin{document}


\title{Documentation for netcomplib}

\maketitle
\tableofcontents
\pagebreak

\listoftodos

\pagebreak

\section{Introduction}

This is a collection of functions that perform hypothesis testing using random edge partitions. \todo{add more to introduction}


\section{Usage Information}
\todo{Write general use case function} -- In general, to simply use the hypothesis testing function, one only needs to input the adjacency matrices (or perhaps arrays). A function for this process hasn't been written yet (only simulation functions that do this a number of times has been written). \\

To use the other pieces of the program for whatever use case: \\

\todo{add no indentation}
Models (eg. SBM or HRGs) can be stored as classes (or a random model can be generated and stored in a class). See the section \todo{link this} to see the documentation on the available classes and the structure. \\

There are some generic functions that work on models, described in the section on Functions \todo{link this}. They perform various tasks, and are almost all used for doing the hypothesis test. \\

\section{Pacakge Detailed Documentation -- functions, classes}
These functions often take in as parameters some complex classes/list structures that are better documented in the package documentation \todo{add here?}. A brief table follows as a guide to function parameters. \\

\begin{tabular}{|l|l|p{8cm}|} \hline
{\bf Parameter Name} & {\bf Object Type} & {\bf Description} \\ 
\hline
NetM & class & Fully specifies a single network model \\ \hline
NetS & class & Fully specifies a random edge partition \\ \hline
param\_list OR pl & list & List of parameters for simulation (or for specifying details for testing) \\ \hline
model\_param & list & List of parameters to generate a single network model or random edge partition\\ \hline 
\end{tabular}




\subsection{Functions}


\subsubsection{Functions}
\begin{itemize} 
\item getNnodes(NetM)\\
This returns the number of nodes in the network
\item getNetType(NetM)\\
This returns the type of network model
\item getEdgeProbMat(NetM, mode))\\
This computes the edge probability matrix\\
Mode = `prob' or `group', will return either a probability matrix or a matrix with ids for group assignment. 
\item sampleNetwork(NetM, Nobs, Nsim)\\
This samples networks from the model
\item extractStruct(NetM)
This converts the network model into a structure object (basically, throws out probability information)
\item computeLik(NetM, adja, loglik = TRUE)\\
This computes the likelihood of the adjacency matrix given the network model. loglik specifies whether we want the log-likelihood or the actual likelihood
\item computeDist(x, y = NULL) \\
Not implemented yet -- Computes the KL distance between two network models (x,y)

\item getNnodes(Net)
\item getNetType(Net)
\item fitModel(NetS, adja)\\
Not implemented -- This fits the model to the adjacency matrix/array, and returns the best fit (MLE) network model. 
\item computePval(NetS, adja1, adja2, Nobs, pl)\\
Returns a matrix of p-values (based on the parameter list) of fitting the model on the given NetS (or, list of NetS's, in which case a list of matrices is returned)
\item computePval\_fast(...)\\
Not implemented -- perhaps write a faster version of this (using global variables, and thus only callable in a very specific manner), as the current code might be slow (due to R's pass by copy as opposed to reference) -- this should end up being written in C++... 
\end{itemize}


\subsection{Classes}

\subsubsection{NetworkModel}
\begin{itemize}
\item NetworkModel (Abbreviated NetM)\\
FIELD -- Nnodes : numeric
\begin{itemize}
\item NetworkModelSBM\\
FIELD -- assign : numeric \\
FIELD -- probmat : matrix\\
\item NetworkModelHRG\\
FIELD -- parents : numeric\\
FIELD -- children : list\\
FIELD -- prob : numeric\\
\item NetworkModelLSM\\
FIELD -- locs : matrix\\
FIELD -- alpha : numeric\\
\item NetworkModelRND\\
FIELD -- counts : numeric\\
FIELD -- prob : numeric\\
FIELD -- ids : list\\
\item NetworkModelPair\\
FIELD -- m1 : NetworkModel\\
FIELD -- m2 : NetworkModel\\
FIELD -- is\_null : logical\\
FIELD -- model\_type : character -- Allowed values:
\begin{itemize}
\item `default'
\item `densitydiff'
\item `correlated' -- not implemented -- {\bf TODO}
\end{itemize}
FIELD -- addl\_param : named list -- Entries: 
\begin{itemize}
\item dd\_param\_add : numeric = added parameter for log-odds addition
\item c\_param\_corr : numeric = parameter for correlation additive factor
\item c\_param\_a : numeric = prob parameter in net1
\item c\_param\_b : numeric = prob parameter in net2
\item c\_param\_names : char = edge group Ids
\end{itemize}
\end{itemize}
\end{itemize}

\subsubsection{NetworkStruct}
\begin{itemize}

\item NetworkStruct (Abbreviated NetS)\\
FIELD -- Nnodes : numeric
\begin{itemize}
\item NetworkStructSBM\\
FIELD -- groups : numeric\\
FIELD -- counts : numeric\\
FIELD -- expand : list\\
FIELD -- correct : numeric\\
\item NetworkStructHRG\\
FIELD -- tree\_list : list\\
FIELD -- expand : list\\
FIELD -- counts : numeric\\
\item NetworkStructRND\\
FIELD -- counts : numeric\\
FIELD -- ids : list\\
\item NetworkStructList\\
FIELD -- models : list\\
\end{itemize}
\end{itemize}









\pagebreak
\section{General Project Details}
\subsection{Version Information}
\subsubsection{v0.1}
This version should have all functions working and well-documented

\subsubsection{v0.2}
Features to potentially add: 
\begin{itemize}
\item Runtime estimation
\item C++ implementation for speed purposes
\item Generic print and summary functions for objects
\end{itemize}



\subsection{Issue Log (for current version)}

\subsubsection*{Issue 4}
Write generic print / summary functions

\subsubsection*{Issue 6}
Update documentation

\subsubsection*{Issue 7}
Plan a C++ version of the code for speed purposes

\subsubsection*{Issue 8}
Write a general use function, that's easy to use

\subsubsection*{Issue 10}
Work on simplifying simulation code: 
\begin{itemize}
\item put model type inside of setmodelparam
\item generalizeble to list of genmodel and list of fit parameters -- do expand.grid of combinations
\item include pvalue adjustment as a setsimparam?
\end{itemize}

\subsubsection*{Issue 11}
in SBM.assign, reset this to 1:num-unique, and resize probmat appropriately

\subsubsection*{Issue 13}
Fix/document verbosity. 
\begin{itemize}
\item param 'verbose' = T/F - simple setting
\item param 'vbset' = (a,b,c) - a = number of functions (and calls) verbose; b = number of functions high-verbose; c = number of filler characters -- (eg +2 per sub function call)
\end{itemize}

\subsubsection*{Issue 14}
Fix recycle\_fitstructs -- true means use max(n\_models). false means use sum(n\_models) -- generate list of indices, and store it in paramlist. when more models were generated; throw away the extras. 

\subsubsection*{Issue 16}
Document computeDfAdj properly -- seems to work, check if it works. 
Figure out appropriate df / distances for new model types (in computeDfAdj and distance computation function

\subsubsection*{Issue 17}
Allow sampling from correlated model

\subsubsection*{Issue 18}
Think about parameter names in fitModel and computePval

\subsubsection*{Issue 19}
Need a standard edge group ordering (need this written down somewhere)

\subsubsection*{Issue 20}
Write tests -- see if new code is reasonable

\subsubsection*{Issue 21}
Verbose logfile? -- so can output to file instead of terminal

\subsubsection*{Issue 22}
Runtime estimation -- run 1\% of the process, and then estimate total runtime

%% \subsection*{Issue }





\pagebreak

\subsection{Documentation Guide}
\begin{verbatim}
Documentation rules -- For parameter listings in function documentation: 
@param <parameter> [<type>] :: <documentation>

<parameter> = parameter name
<documentation> = verbal description of usage of parameter
<type> = classification of parameter

Inside of <type>: 
, will separate information about parameters: eg. 
LEN = (length of vector), can also be expressed as vector[LEN]
DIM = (dimensions of matrix, etc.)
; will separate information about values the parameter takes on
DEFAULT = (default value)
ALLOWED = (possibly a vector of allowed values)
OR = separates different possible types of paraemterization (or output)
- = identifies subtype of more general type of parameter

Example: 
[vector-int, LEN = 2; DEFAULT = c(1,2)] :: some description
list, matrix, array, vector
int, double, char, logical

For documentation of classes: (S3 classes)
an \itemize with items as follows: 
\item <name> -- [<mode description as above>] :: <documentation>

\end{verbatim}


\section{Specific Implementation Details}


\end{document}

